\documentclass{beamer}

\usepackage{graphicx}

% Title and Author
\title{Rapid Word Learning Under Uncertainty via Cross-Situational Statistics}
\author{Based on Yu \& Smith (2007)}
\date{}

\begin{document}
	
	% Title Slide
	\begin{frame}
		\titlepage
	\end{frame}
	
	% Slide 1: Introduction
	\begin{frame}{Introduction}
		\textbf{Main Question:}
		\begin{itemize}
			\item How can humans learn word-referent pairs in ambiguous contexts?
			\item Can cross-situational statistics help solve the indeterminacy problem?
		\end{itemize}
		\vspace{0.3cm}
		\textbf{Why It's Interesting:}
		\begin{itemize}
			\item Real-world word learning involves many ambiguous contexts.
			\item Challenges the traditional focus on single-trial constraints.
		\end{itemize}
		\vspace{0.3cm}
		\textbf{Why We Care:}
		\begin{itemize}
			\item Explores mechanisms behind robust vocabulary development in children and adults.
		\end{itemize}
	\end{frame}
	
	% Slide 2: Background Literature
	\begin{frame}{Background Literature}
		\textbf{Previous Research:}
		\begin{itemize}
			\item Fast mapping through attentional, social, and linguistic cues in single trials (e.g., Baldwin, 1993; Smith, 2000).
			\item Computational models suggest cross-situational learning is plausible (e.g., Siskind, 1996; Vogt \& Smith, 2005).
		\end{itemize}
		\vspace{0.3cm}
		\textbf{Gap:}
		\begin{itemize}
			\item Limited systematic investigation of human learners' ability to use cross-situational statistics.
		\end{itemize}
	\end{frame}
	
	% Slide 3: Methods - Overview
	\begin{frame}{Methods: Overview}
		\textbf{Participants:}
		\begin{itemize}
			\item 38 university students.
		\end{itemize}
		\vspace{0.3cm}
		\textbf{Stimuli:}
		\begin{itemize}
			\item Pictures of uncommon objects paired with pseudowords.
		\end{itemize}
		\vspace{0.3cm}
		\textbf{Task:}
		\begin{itemize}
			\item Identify word-referent mappings across ambiguous trials.
		\end{itemize}
	\end{frame}
	
	% Slide 4: Methods - Experimental Design
	\begin{frame}{Methods: Experimental Design}
		\textbf{Learning Conditions:}
		\begin{itemize}
			\item \textbf{2×2 Condition:} 2 words and 2 objects (4 associations per trial).
			\item \textbf{3×3 Condition:} 3 words and 3 objects (9 associations per trial).
			\item \textbf{4×4 Condition:} 4 words and 4 objects (16 associations per trial).
		\end{itemize}
		\vspace{0.3cm}
		\textbf{Procedure:}
		\begin{itemize}
			\item Visual objects displayed on a screen; pseudowords played via audio.
			\item Forced-choice test after training: Match words to referents.
		\end{itemize}
	\end{frame}
\end{document}